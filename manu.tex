%% bare_conf.tex
%% V1.3
%% 2007/01/11
%% by Michael Shell
%% See:
%% http://www.michaelshell.org/
%% for current contact information.
%%
%% This is a skeleton file demonstrating the use of IEEEtran.cls
%% (requires IEEEtran.cls version 1.7 or later) with an IEEE conference paper.
%%
%% Support sites:
%% http://www.michaelshell.org/tex/ieeetran/
%% http://www.ctan.org/tex-archive/macros/latex/contrib/IEEEtran/
%% and
%% http://www.ieee.org/

%%*************************************************************************
%% Legal Notice:
%% This code is offered as-is without any warranty either expressed or
%% implied; without even the implied warranty of MERCHANTABILITY or
%% FITNESS FOR A PARTICULAR PURPOSE! 
%% User assumes all risk.
%% In no event shall IEEE or any contributor to this code be liable for
%% any damages or losses, including, but not limited to, incidental,
%% consequential, or any other damages, resulting from the use or misuse
%% of any information contained here.
%%
%% All comments are the opinions of their respective authors and are not
%% necessarily endorsed by the IEEE.
%%
%% This work is distributed under the LaTeX Project Public License (LPPL)
%% ( http://www.latex-project.org/ ) version 1.3, and may be freely used,
%% distributed and modified. A copy of the LPPL, version 1.3, is included
%% in the base LaTeX documentation of all distributions of LaTeX released
%% 2003/12/01 or later.
%% Retain all contribution notices and credits.
%% ** Modified files should be clearly indicated as such, including  **
%% ** renaming them and changing author support contact information. **
%%
%% File list of work: IEEEtran.cls, IEEEtran_HOWTO.pdf, bare_adv.tex,
%%                    bare_conf.tex, bare_jrnl.tex, bare_jrnl_compsoc.tex
%%*************************************************************************

% *** Authors should verify (and, if needed, correct) their LaTeX system  ***
% *** with the testflow diagnostic prior to trusting their LaTeX platform ***
% *** with production work. IEEE's font choices can trigger bugs that do  ***
% *** not appear when using other class files.                            ***
% The testflow support page is at:
% http://www.michaelshell.org/tex/testflow/



% Note that the a4paper option is mainly intended so that authors in
% countries using A4 can easily print to A4 and see how their papers will
% look in print - the typesetting of the document will not typically be
% affected with changes in paper size (but the bottom and side margins will).
% Use the testflow package mentioned above to verify correct handling of
% both paper sizes by the user's LaTeX system.
%
% Also note that the "draftcls" or "draftclsnofoot", not "draft", option
% should be used if it is desired that the figures are to be displayed in
% draft mode.
%
\documentclass[conference]{IEEEtran}
% Add the compsoc option for Computer Society conferences.
%
% If IEEEtran.cls has not been installed into the LaTeX system files,
% manually specify the path to it like:
% \documentclass[conference]{../sty/IEEEtran}





% Some very useful LaTeX packages include:
% (uncomment the ones you want to load)


% *** MISC UTILITY PACKAGES ***
%
%\usepackage{ifpdf}
% Heiko Oberdiek's ifpdf.sty is very useful if you need conditional
% compilation based on whether the output is pdf or dvi.
% usage:
% \ifpdf
%   % pdf code
% \else
%   % dvi code
% \fi
% The latest version of ifpdf.sty can be obtained from:
% http://www.ctan.org/tex-archive/macros/latex/contrib/oberdiek/
% Also, note that IEEEtran.cls V1.7 and later provides a builtin
% \ifCLASSINFOpdf conditional that works the same way.
% When switching from latex to pdflatex and vice-versa, the compiler may
% have to be run twice to clear warning/error messages.






% *** CITATION PACKAGES ***
%
%\usepackage{cite}
% cite.sty was written by Donald Arseneau
% V1.6 and later of IEEEtran pre-defines the format of the cite.sty package
% \cite{} output to follow that of IEEE. Loading the cite package will
% result in citation numbers being automatically sorted and properly
% "compressed/ranged". e.g., [1], [9], [2], [7], [5], [6] without using
% cite.sty will become [1], [2], [5]--[7], [9] using cite.sty. cite.sty's
% \cite will automatically add leading space, if needed. Use cite.sty's
% noadjust option (cite.sty V3.8 and later) if you want to turn this off.
% cite.sty is already installed on most LaTeX systems. Be sure and use
% version 4.0 (2003-05-27) and later if using hyperref.sty. cite.sty does
% not currently provide for hyperlinked citations.
% The latest version can be obtained at:
% http://www.ctan.org/tex-archive/macros/latex/contrib/cite/
% The documentation is contained in the cite.sty file itself.






% *** GRAPHICS RELATED PACKAGES ***
%
\ifCLASSINFOpdf
  % \usepackage[pdftex]{graphicx}
  % declare the path(s) where your graphic files are
  % \graphicspath{{../pdf/}{../jpeg/}}
  % and their extensions so you won't have to specify these with
  % every instance of \includegraphics
  % \DeclareGraphicsExtensions{.pdf,.jpeg,.png}
\else
  % or other class option (dvipsone, dvipdf, if not using dvips). graphicx
  % will default to the driver specified in the system graphics.cfg if no
  % driver is specified.
  % \usepackage[dvips]{graphicx}
  % declare the path(s) where your graphic files are
  % \graphicspath{{../eps/}}
  % and their extensions so you won't have to specify these with
  % every instance of \includegraphics
  % \DeclareGraphicsExtensions{.eps}
\fi
% graphicx was written by David Carlisle and Sebastian Rahtz. It is
% required if you want graphics, photos, etc. graphicx.sty is already
% installed on most LaTeX systems. The latest version and documentation can
% be obtained at: 
% http://www.ctan.org/tex-archive/macros/latex/required/graphics/
% Another good source of documentation is "Using Imported Graphics in
% LaTeX2e" by Keith Reckdahl which can be found as epslatex.ps or
% epslatex.pdf at: http://www.ctan.org/tex-archive/info/
%
% latex, and pdflatex in dvi mode, support graphics in encapsulated
% postscript (.eps) format. pdflatex in pdf mode supports graphics
% in .pdf, .jpeg, .png and .mps (metapost) formats. Users should ensure
% that all non-photo figures use a vector format (.eps, .pdf, .mps) and
% not a bitmapped formats (.jpeg, .png). IEEE frowns on bitmapped formats
% which can result in "jaggedy"/blurry rendering of lines and letters as
% well as large increases in file sizes.
%
% You can find documentation about the pdfTeX application at:
% http://www.tug.org/applications/pdftex





% *** MATH PACKAGES ***
%
%\usepackage[cmex10]{amsmath}
% A popular package from the American Mathematical Society that provides
% many useful and powerful commands for dealing with mathematics. If using
% it, be sure to load this package with the cmex10 option to ensure that
% only type 1 fonts will utilized at all point sizes. Without this option,
% it is possible that some math symbols, particularly those within
% footnotes, will be rendered in bitmap form which will result in a
% document that can not be IEEE Xplore compliant!
%
% Also, note that the amsmath package sets \interdisplaylinepenalty to 10000
% thus preventing page breaks from occurring within multiline equations. Use:
%\interdisplaylinepenalty=2500
% after loading amsmath to restore such page breaks as IEEEtran.cls normally
% does. amsmath.sty is already installed on most LaTeX systems. The latest
% version and documentation can be obtained at:
% http://www.ctan.org/tex-archive/macros/latex/required/amslatex/math/





% *** SPECIALIZED LIST PACKAGES ***
%
\usepackage{algorithmic}
% algorithmic.sty was written by Peter Williams and Rogerio Brito.
% This package provides an algorithmic environment fo describing algorithms.
% You can use the algorithmic environment in-text or within a figure
% environment to provide for a floating algorithm. Do NOT use the algorithm
% floating environment provided by algorithm.sty (by the same authors) or
% algorithm2e.sty (by Christophe Fiorio) as IEEE does not use dedicated
% algorithm float types and packages that provide these will not provide
% correct IEEE style captions. The latest version and documentation of
% algorithmic.sty can be obtained at:
% http://www.ctan.org/tex-archive/macros/latex/contrib/algorithms/
% There is also a support site at:
% http://algorithms.berlios.de/index.html
% Also of interest may be the (relatively newer and more customizable)
% algorithmicx.sty package by Szasz Janos:
% http://www.ctan.org/tex-archive/macros/latex/contrib/algorithmicx/




% *** ALIGNMENT PACKAGES ***
%
%\usepackage{array}
% Frank Mittelbach's and David Carlisle's array.sty patches and improves
% the standard LaTeX2e array and tabular environments to provide better
% appearance and additional user controls. As the default LaTeX2e table
% generation code is lacking to the point of almost being broken with
% respect to the quality of the end results, all users are strongly
% advised to use an enhanced (at the very least that provided by array.sty)
% set of table tools. array.sty is already installed on most systems. The
% latest version and documentation can be obtained at:
% http://www.ctan.org/tex-archive/macros/latex/required/tools/


\usepackage{mdwmath}
\usepackage{mdwtab}
% Also highly recommended is Mark Wooding's extremely powerful MDW tools,
% especially mdwmath.sty and mdwtab.sty which are used to format equations
% and tables, respectively. The MDWtools set is already installed on most
% LaTeX systems. The lastest version and documentation is available at:
% http://www.ctan.org/tex-archive/macros/latex/contrib/mdwtools/


% IEEEtran contains the IEEEeqnarray family of commands that can be used to
% generate multiline equations as well as matrices, tables, etc., of high
% quality.


%\usepackage{eqparbox}
% Also of notable interest is Scott Pakin's eqparbox package for creating
% (automatically sized) equal width boxes - aka "natural width parboxes".
% Available at:
% http://www.ctan.org/tex-archive/macros/latex/contrib/eqparbox/





% *** SUBFIGURE PACKAGES ***
%\usepackage[tight,footnotesize]{subfigure}
% subfigure.sty was written by Steven Douglas Cochran. This package makes it
% easy to put subfigures in your figures. e.g., "Figure 1a and 1b". For IEEE
% work, it is a good idea to load it with the tight package option to reduce
% the amount of white space around the subfigures. subfigure.sty is already
% installed on most LaTeX systems. The latest version and documentation can
% be obtained at:
% http://www.ctan.org/tex-archive/obsolete/macros/latex/contrib/subfigure/
% subfigure.sty has been superceeded by subfig.sty.



%\usepackage[caption=false]{caption}
\usepackage[font=footnotesize]{subfig}
% subfig.sty, also written by Steven Douglas Cochran, is the modern
% replacement for subfigure.sty. However, subfig.sty requires and
% automatically loads Axel Sommerfeldt's caption.sty which will override
% IEEEtran.cls handling of captions and this will result in nonIEEE style
% figure/table captions. To prevent this problem, be sure and preload
% caption.sty with its "caption=false" package option. This is will preserve
% IEEEtran.cls handing of captions. Version 1.3 (2005/06/28) and later 
% (recommended due to many improvements over 1.2) of subfig.sty supports
% the caption=false option directly:
%\usepackage[caption=false,font=footnotesize]{subfig}
%
% The latest version and documentation can be obtained at:
% http://www.ctan.org/tex-archive/macros/latex/contrib/subfig/
% The latest version and documentation of caption.sty can be obtained at:
% http://www.ctan.org/tex-archive/macros/latex/contrib/caption/




% *** FLOAT PACKAGES ***
%
%\usepackage{fixltx2e}
% fixltx2e, the successor to the earlier fix2col.sty, was written by
% Frank Mittelbach and David Carlisle. This package corrects a few problems
% in the LaTeX2e kernel, the most notable of which is that in current
% LaTeX2e releases, the ordering of single and double column floats is not
% guaranteed to be preserved. Thus, an unpatched LaTeX2e can allow a
% single column figure to be placed prior to an earlier double column
% figure. The latest version and documentation can be found at:
% http://www.ctan.org/tex-archive/macros/latex/base/



%\usepackage{stfloats}
% stfloats.sty was written by Sigitas Tolusis. This package gives LaTeX2e
% the ability to do double column floats at the bottom of the page as well
% as the top. (e.g., "\begin{figure*}[!b]" is not normally possible in
% LaTeX2e). It also provides a command:
%\fnbelowfloat
% to enable the placement of footnotes below bottom floats (the standard
% LaTeX2e kernel puts them above bottom floats). This is an invasive package
% which rewrites many portions of the LaTeX2e float routines. It may not work
% with other packages that modify the LaTeX2e float routines. The latest
% version and documentation can be obtained at:
% http://www.ctan.org/tex-archive/macros/latex/contrib/sttools/
% Documentation is contained in the stfloats.sty comments as well as in the
% presfull.pdf file. Do not use the stfloats baselinefloat ability as IEEE
% does not allow \baselineskip to stretch. Authors submitting work to the
% IEEE should note that IEEE rarely uses double column equations and
% that authors should try to avoid such use. Do not be tempted to use the
% cuted.sty or midfloat.sty packages (also by Sigitas Tolusis) as IEEE does
% not format its papers in such ways.





% *** PDF, URL AND HYPERLINK PACKAGES ***
%
\usepackage{url}
% url.sty was written by Donald Arseneau. It provides better support for
% handling and breaking URLs. url.sty is already installed on most LaTeX
% systems. The latest version can be obtained at:
% http://www.ctan.org/tex-archive/macros/latex/contrib/misc/
% Read the url.sty source comments for usage information. Basically,
% \url{my_url_here}.





% *** Do not adjust lengths that control margins, column widths, etc. ***
% *** Do not use packages that alter fonts (such as pslatex).         ***
% There should be no need to do such things with IEEEtran.cls V1.6 and later.
% (Unless specifically asked to do so by the journal or conference you plan
% to submit to, of course. )


% correct bad hyphenation here
\hyphenation{op-tical net-works semi-conduc-tor}


\begin{document}
%
% paper title
% can use linebreaks \\ within to get better formatting as desired
\title{P2P CDN}

\author{\IEEEauthorblockN{Mohamad Dikshie Fauzie\IEEEauthorrefmark{1} \quad
Achmad Husni Thamrin\IEEEauthorrefmark{1} \quad
Jun Murai\IEEEauthorrefmark{2}}
\IEEEauthorblockA{\IEEEauthorrefmark{1}Graduate School of Media and Governance} \quad
\IEEEauthorblockA{\IEEEauthorrefmark{2}Faculty of Environment and Information Studies\\ 
Keio University, 252-0882 Kanagawa, Japan \\
dikshie@sfc.wide.ad.jp \quad\quad husni@ai3.net \quad\quad jun@wide.ad.jp}
}




% use for special paper notices
%\IEEEspecialpapernotice{(Invited Paper)}




% make the title area
\maketitle


\begin{abstract}
%\boldmath
The abstract goes here.
\end{abstract}
% IEEEtran.cls defaults to using nonbold math in the Abstract.
% This preserves the distinction between vectors and scalars. However,
% if the conference you are submitting to favors bold math in the abstract,
% then you can use LaTeX's standard command \boldmath at the very start
% of the abstract to achieve this. Many IEEE journals/conferences frown on
% math in the abstract anyway.

% no keywords




% For peer review papers, you can put extra information on the cover
% page as needed:
% \ifCLASSOPTIONpeerreview
% \begin{center} \bfseries EDICS Category: 3-BBND \end{center}
% \fi
%
% For peerreview papers, this IEEEtran command inserts a page break and
% creates the second title. It will be ignored for other modes.
\IEEEpeerreviewmaketitle



\section{Introduction}
Streaming content, especially video, represents a significant fraction of the traffic volume on the Internet, and it has become a standard practice to deliver this type of content using Content Delivery Networks (CDNs) such as Akamai and Limelight for better scaling and quality of experience for the end users. 
For example, YouTube uses Google cache and MTV uses Akamai in their operations.

With the spread of broadband Internet access at a reasonable flat monthly rate, users are connected to the Internet 24 hours a day and they can download and share multimedia content. P2P (peer to peer) applications are also widely deployed. 
In China, P2P is very popular; we see many P2P applications from China such as PPLive, PPStream, UUSe, Xunlei, etc. \cite{Vu:2010:UOC:1865106.1865115}. 
Some news broadcasters also rely on P2P technology to deliver popular live events. 
For example, CNN uses the Octoshape \cite{octoshape} solution that enables their broadcast to scale and offer good video quality as the number of users increases.

From the Internet provider point of view, the presence of so many always-on users suggests that it is possible to delegate a portion of computing, storage and networking tasks to the users, thus creating P2P networks where users can share files and multimedia content. 
Starting from file sharing protocols, P2P architectures have evolved toward video on demand and support for live events.

A P2P based architecture usually requires a sufficient number of nodes supplying the data (seeders) to start the distribution process among the joining peers.  
A peer usually offers a low outbound streaming rate due to the traditional asymmetrical DSL home connectivity and hence multiple peers must jointly stream contents to a requesting peer (leecher).  
The decentralized, uncoordinated operation implies that scaling to high number of peers comes with side effects.  
Typical problems of a P2P streaming architecture are low stream quality with undesirable disruptions, resource unfairness due to heterogeneous peer resources, and high startup delay.  
Moreover, current P2P applications are not aware of the underlying network and may conflict with the ISP routing policies and business model.

A number of P2P streaming applications have been designed, analyzed and deployed, attracting a significant number of users.  
Research studies and deployment experiences have both demonstrated that P2P is a promising solution in terms of scalability and deployment costs.  
On the other hand, the heterogeneous nature and unstable behavior of the peers contributing bandwidth and computational resources, along with the networking issues, affect the user experience and limit the commercial success of P2P video streaming applications.
Alternatively, video contents can be efficiently distributed on services offered by managed network architectures and CDN companies.
The major issues of CDN are high deployment cost and good but not unlimited scalability in the long term.  
Given the complementary features of P2P and CDN, in recent years some hybrid solutions have been proposed \cite{Huang:2008:UHC:1496046.1496064,4772628,Yin:2009:DDH:1631272.1631279} to take the best of both approaches.

Broadband network access helps P2P applications to perform better. xDSL networks are deployed worldwide, and in some countries, such as Japan, even higher bandwidth fiber to the home (FTTH) already exceeds DSL in market penetration. 
In the coming years, network operators throughout the world will massively deploy FTTH. 
As access bandwidth increases, P2P systems may become more efficient since a peer can contribute much more.

Typically, each end user device is involved directly in the P2P swarm, both to receive the benefits of P2P and to serve others.  
In such cases, the installation of P2P software in every end device is necessary and the user is directly involved in the content's swarm.
Under such conditions, the disposition of peers may result in unstable behavior and the swarm can be affected by the rapid and frequent disconnections that are common for mobile devices.  
Furthermore, users' devices usually can contribute to the swarm only with limited upload bandwidth.  
In addition, techniques developed to select P2P neighboring peers are often unfriendly toward the ISP's routing policies.

Different topologies have been proposed in the literature, such as those where the collaborative mechanism for content distribution is
created among more stable devices such as residential gateways.  
The residential gateway (i.e., a home gateway placed in at the user's premises, serving several terminals within the home network but directly managed by ISP) is considered the central entity for a managed P2P infrastructure \cite{Misra:2010:IPS:1811099.1811064,Cha:2008:NTP:1855641.1855646}.
Running on more stable and powerful devices, each gateway peer can contribute more bandwidth to the content swarm compared to the traditional end-user P2P systems.  Peer selection procedures can be managed directly by the ISP, with the goal of avoiding the traversal of multiple nodes across ISP boundaries.  
Since P2P traffic is now decreasing and moving to the cloud \cite{Labovitz:2010:IIT:2043164.1851194}, there is plenty of headroom for the ISP to use the gateway in a peer-assisted CDN, and the always-on nature of the gateway makes it the perfect device to run peer-assisted applications.  
ISPs may even be willing to give rebates to users who allow their gateways to be used, since the ISP benefits from incorporating the gateway into their CDN.  With the growing interest to interconnect CDNs \cite{cdni,oceanproject}, this architecture can benefit the ISP.

In Peer assisted CDN, users can download content from CDN nodes from or other users or peers. 
A user may cache the content after download to serve requests from other users. 
Due to the complexity of the behavior of peers, the process should be done in the home gateway user where the ISP can control it.

In this paper, we present. 
In summary this paper makes the following contributions:
Finally, we present our conclusions in section \ref{conclusion}




\section{Related Work}\label{relatedwork}
Content Distribution Networks with peer assist have been successfully deployed on the Internet, such as Akamai cite{Zhao:2013:PCD:2504730.2504752}, \cite{Huang:2008:UHC:1496046.1496064} and LiveSky \cite{Yin:2010:LEC:1823746.1823750}.  
The authors of \cite{Zhao:2013:PCD:2504730.2504752} examine the risks and benefits of peer-assisted content distribution in Akamai and measure the effectiveness of its peer-assisted. 
The authors of \cite{Huang:2008:UHC:1496046.1496064} conclude from two real world traces that hybrid CDN-P2P can significantly reduce the cost of content distribution and can scale to cope with the exponential growth of Internet video content.  
Yin et al. \cite{Yin:2010:LEC:1823746.1823750} described commercial operation of a peer-assisted CDN in China.  
LiveSky solved several challenges in the system design, such as dynamic resource scaling of P2P, low startup latency, ease of P2P integration with the existing CDN infrastructure, and network friendliness and upload fairness in the P2P operation.  
Xu et al.\cite{DBLP:journals/corr/abs-1212-4915} using game theory, showed that the right cooperative profit distribution of P2P can help the ISP to maximize the utility.  
Their model can easily be implemented in the context of current Internet economic settlements.  
Misra et al.\cite{Misra:2010:IPS:1811099.1811064} also mentioned the importance of P2P architecture to support content delivery networks.
The authors use cooperative game theory to formulate simple compensation rules for users who run P2P to support content delivery networks.

The idea of telco- or ISP-managed CDN has been proposed in recent years.  
The complexity of the CDN business encourage telcos and ISPs to manage their own CDN, rather than allow others to run CDNs on their networks.  
It has been shown that it is cost effective \cite{federation}\cite{norton2011internet}. 
Kamiyama et al. \cite{NoriakiKAMIYAMA2013} proposed optimally ISP operated CDN.
Kamiyama et al. mentioned that, in order to deliver large and rich Internet content to users, ISPs need to put their CDNs in data centers.  
The locations are limited while the storage is large, making this solution effective, using optimum placement algorithm based on real ISP network topologies.  
The authors found that inserting a CDN into an ISP's ladder-type network is effective in reducing the hop count, thus reduce total link cost.  
Cisco has initiated an effort to connect telco- or ISP-managed CDNs to each other, to form a CDN federation \cite{federation} using open standards \cite{cdni}.  
They argue that the current CDN architecture is not close enough to the users and ISPs can fill this position.

The idea of utilizing the user's computation power to support ISP operation is not new.  
The Figaro project \cite{figaro} proposed residential gateway as an integrator of different networks and services, becoming an Internet-wide distributed content management for a proposed future Internet architecture \cite{figaro}.  
Cha et al.,\cite{Cha:2008:NTP:1855641.1855646} performed trace analysis and found that an IPTV architecture powered by P2P can handle a much larger number of channels, with limited demand for infrastructure compare to IP multicast.  
Jiang et al. \cite{Jiang:2012:OMD:2413176.2413193} proposed scalable and adaptive content replication and request routing for CDN servers located in users' home gateways.  
Maki et al.\cite{NaoyaMAKI2012} propose traffic engineering for peer-assisted CDN to control the behavior of clients, and present a solution for optimizing the selection of content files.
Mathieu et al., \cite{6249305} are using data gathered from France telecom network to calculate reduction of network load if customers are employed as peer-assisted content delivery.
Our work is same in system model architecture which uses different level of topologies  (Level-0 and Level-1), where in Mathieu et al., \cite{6249305} the authors use different names, which are regional and national. 
We emphasize that compare to  Mathieu et al., \cite{6249305} our work provides completely different approach.
Mathieu et al., \cite{6249305} work mostly based on empirical data from France telecom company thus the authors can directly calculate network load caused by video traffic and calculate network load reduction if peer-assisted is employed on customers side,  while our study focus on mathematical model of different admission policies for peers to join peer-assisted CDN and ISP payoff can get from employing peer-assisted CDN.


\section{Characterizing Internet VoD Popularity}\label{popularity}
Before analyzing the system description and video caching, we first examine some characteristics of Internet VoD services.
The studies of content popularity evolution are mostly considered in short time periods.
Borghol et al., \cite{Borghol:2011:CMP:2039452.2039717} measure the evolution of content popularity in long periods (36 weeks) in which view count statistics of Youtube. 
We also did same thing as Borghol et al., \cite{Borghol:2011:CMP:2039452.2039717} for two months from October-November 2013.
Our measurement is additional to Borghol's datasets and become basis of our popularity analysis.







\section{System Description}\label{systemdescription}
In this paper, we consider a peer-assisted CDN system. 
In such a system there are two main components: (1) the CDN server which at a minimum consist content delivery platform and control plane platform. 
(2) the client system which requests and downloads the videos.
In addition, clients form a self-organized P2P overlay network.

\subsection{Collection and Estimation}

\subsection{Per cache strategy}


\section{Evaluation}\label{evaluation}

% An example of a floating figure using the graphicx package.
% Note that \label must occur AFTER (or within) \caption.
% For figures, \caption should occur after the \includegraphics.
% Note that IEEEtran v1.7 and later has special internal code that
% is designed to preserve the operation of \label within \caption
% even when the captionsoff option is in effect. However, because
% of issues like this, it may be the safest practice to put all your
% \label just after \caption rather than within \caption{}.
%
% Reminder: the "draftcls" or "draftclsnofoot", not "draft", class
% option should be used if it is desired that the figures are to be
% displayed while in draft mode.
%
%\begin{figure}[!t]
%\centering
%\includegraphics[width=2.5in]{myfigure}
% where an .eps filename suffix will be assumed under latex, 
% and a .pdf suffix will be assumed for pdflatex; or what has been declared
% via \DeclareGraphicsExtensions.
%\caption{Simulation Results}
%\label{fig_sim}
%\end{figure}

% Note that IEEE typically puts floats only at the top, even when this
% results in a large percentage of a column being occupied by floats.


% An example of a double column floating figure using two subfigures.
% (The subfig.sty package must be loaded for this to work.)
% The subfigure \label commands are set within each subfloat command, the
% \label for the overall figure must come after \caption.
% \hfil must be used as a separator to get equal spacing.
% The subfigure.sty package works much the same way, except \subfigure is
% used instead of \subfloat.
%
%\begin{figure*}[!t]
%\centerline{\subfloat[Case I]\includegraphics[width=2.5in]{subfigcase1}%
%\label{fig_first_case}}
%\hfil
%\subfloat[Case II]{\includegraphics[width=2.5in]{subfigcase2}%
%\label{fig_second_case}}}
%\caption{Simulation results}
%\label{fig_sim}
%\end{figure*}
%
% Note that often IEEE papers with subfigures do not employ subfigure
% captions (using the optional argument to \subfloat), but instead will
% reference/describe all of them (a), (b), etc., within the main caption.


% An example of a floating table. Note that, for IEEE style tables, the 
% \caption command should come BEFORE the table. Table text will default to
% \footnotesize as IEEE normally uses this smaller font for tables.
% The \label must come after \caption as always.
%
%\begin{table}[!t]
%% increase table row spacing, adjust to taste
%\renewcommand{\arraystretch}{1.3}
% if using array.sty, it might be a good idea to tweak the value of
% \extrarowheight as needed to properly center the text within the cells
%\caption{An Example of a Table}
%\label{table_example}
%\centering
%% Some packages, such as MDW tools, offer better commands for making tables
%% than the plain LaTeX2e tabular which is used here.
%\begin{tabular}{|c||c|}
%\hline
%One & Two\\
%\hline
%Three & Four\\
%\hline
%\end{tabular}
%\end{table}


% Note that IEEE does not put floats in the very first column - or typically
% anywhere on the first page for that matter. Also, in-text middle ("here")
% positioning is not used. Most IEEE journals/conferences use top floats
% exclusively. Note that, LaTeX2e, unlike IEEE journals/conferences, places
% footnotes above bottom floats. This can be corrected via the \fnbelowfloat
% command of the stfloats package.


\section{Conclusion and Future Work}\label{conclusion}
This paper presents a scheme for a ISP managed peer-assisted CDN model that 
Some areas of improvement that we have identified for future are:
We are also very interested to include energy trade off this peer-assisted CDN architecture in order to know how much energy saving by ISP and how much increase of energy at users home gateway side in this architecture.





% conference papers do not normally have an appendix


% use section* for acknowledgement
\section*{Acknowledgment}

The authors would like to thank...





% trigger a \newpage just before the given reference
% number - used to balance the columns on the last page
% adjust value as needed - may need to be readjusted if
% the document is modified later
%\IEEEtriggeratref{8}
% The "triggered" command can be changed if desired:
%\IEEEtriggercmd{\enlargethispage{-5in}}

% references section

% can use a bibliography generated by BibTeX as a .bbl file
% BibTeX documentation can be easily obtained at:
% http://www.ctan.org/tex-archive/biblio/bibtex/contrib/doc/
% The IEEEtran BibTeX style support page is at:
% http://www.michaelshell.org/tex/ieeetran/bibtex/
%\bibliographystyle{IEEEtran}
% argument is your BibTeX string definitions and bibliography database(s)
%\bibliography{IEEEabrv,../bib/paper}
%
% <OR> manually copy in the resultant .bbl file
% set second argument of \begin to the number of references
% (used to reserve space for the reference number labels box)
\bibliographystyle{IEEEtran}
\bibliography{manu}



% that's all folks
\end{document}


